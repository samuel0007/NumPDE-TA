%Header for the instruction files


%%%%%%%%%%%%%%%%%%%%%%%%%%%%%%%%%%%%%%%%%%%% PACKAGES %%%%%%%%%%%%%%%%%%%%%%%%%%%%%%%%%%%%%%%%%%%%%%%%%%%%%%%%%

%euopean version of article, because paper format is A4
\documentclass{scrartcl}

\usepackage[a4paper, left=5mm, right=10cm, top=1cm, marginparwidth=9cm]{geometry}

%UTF-8, because want decent encoding
\usepackage[utf8]{inputenc}

%\usepackage[ngerman]{babel}

%for ae ue oe
%\usepackage{ae}
\usepackage[T1]{fontenc}

%better monospaced font
\usepackage[scaled=0.8]{beramono}

%to have math symbols available
\usepackage{amsmath,amssymb,euscript}

%pretty enumerations
\usepackage{enumerate}

\usepackage{physics}

\usepackage{mathtools}

%pictures 
\usepackage{graphicx}
%\DeclareGraphicsRule{.tif}{png}{.png}{`convert #1 `dirname #1`/`basename #1 .tif`.png}

%listings are a nice way to display source code
\usepackage{listings, color}
\usepackage[table]{xcolor}

%define another version of green in addition to the 16 tex colors
\definecolor{mygreen}{rgb}{0.133,0.545,0.133}

%for weblinks
\usepackage[hidelinks]{hyperref}

\usepackage{mdframed}

\usepackage{tocloft}

\usepackage{marginnote}

\usepackage{tabularx}

\usepackage{ textcomp }

%%%%%%%%%%%%%%%%%%%%%%%%%%%%%%%%%%%%%%%%%%%% COMMANDS AND SETTINGS %%%%%%%%%%%%%%%%%%%%%%%%%%%%%%%%%%%%%%%%%%%%


%noindent, because text passages are very short to begin with
\setlength\parindent{0pt}

%instead the distance between lines is a bit bigger, if a paragraph ends
%this is a default value, it is sometimes changed locally, for example whenever itemize is used
%\setlength{\parskip}{1.0cm plus4mm minus3mm}

%use these commands when you want to create a list of points using itemize
%the distance between paragraphs defined above makes lists look very bad, use /nogap before creating a list and \gap after to restore settings
\newcommand{\nogap}{}%\setlength{\parskip}{0.0cm}}
\newcommand{\gap}{}%\setlength{\parskip}{1.0cm plus4mm minus3mm}}


%command that configures the code listings to be in a box with line numbers to the left
%it also makes sure that there is space between the last paragraph and the source code, so make sure you use this command before every 
%listing of a complete program
\newcommand{\numbersleft}{ 
\setlength{\parskip}{0cm}
{\scriptsize \mbox{}} \\ 
\setlength{\parskip}{1.0cm plus4mm minus3mm}
\lstset{ %
  language=C++, % choose the language of the code
  basicstyle=\footnotesize\ttfamily, % the size of the fonts that are used for the code
  numbers=left, % where to put the line-numbers
  numberstyle=\footnotesize\ttfamily\color[rgb]{0.6,0.6,0.6}, % the size of the fonts that are used for the line-numbers
  stepnumber=1, % the step between two line-numbers. If it's 1 each line
  xleftmargin=1mm,
  xrightmargin=1mm,
  % will be numbered
  numbersep=10pt, % how far the line-numbers are from the code
  backgroundcolor=\color{white}, % choose the background color. You must add \usepackage{color}
  showspaces=false, % show spaces adding particular underscores
  showstringspaces=false, % underline spaces within strings
  showtabs=false, % show tabs within strings adding particular underscores
  %frame=l, % adds a frame around the code
  frame=single,
  tabsize=4, % sets default tabsize to 2 spaces
  breaklines=true, % sets automatic line breaking
  breakatwhitespace=false, % sets if automatic breaks should only happen at whitespace
  % also try caption instead of title
  escapeinside={\%*}{*)}, % if you want to add a comment within your code
  morekeywords={*,...}, % if you want to add more keywords to the set
  keywordstyle=\color[rgb]{0,0,1},
  commentstyle=\color[rgb]{0.133,0.545,0.133}\textit,
  stringstyle=\color[rgb]{0.627,0.126,0.941},
}}

%command that configures the code listings to be without a box and without line numbers to the left
%it also makes sure that there is space between the last paragraph and the source code, so make sure you use this command before every listing which
%is just a code snippet
\newcommand{\nonumbers}{
\setlength{\parskip}{0cm}
{\scriptsize \mbox{}} \\ 
\setlength{\parskip}{1.0cm plus4mm minus3mm}
\lstset{ %
  language=C++, % choose the language of the code
  basicstyle=\footnotesize\ttfamily, % the size of the fonts that are used for the code
  numbers=none, % where to put the line-numbers
  numberstyle=\footnotesize\ttfamily\color[rgb]{0.6,0.6,0.6}, % the size of the fonts that are used for the line-numbers
  stepnumber=1, % the step between two line-numbers. If it's 1 each line
  xleftmargin=4mm,
  % will be numbered
  numbersep=5pt, % how far the line-numbers are from the code
  backgroundcolor=\color{white}, % choose the background color. You must add \usepackage{color}
  showspaces=false, % show spaces adding particular underscores
  showstringspaces=false, % underline spaces within strings
  showtabs=false, % show tabs within strings adding particular underscores
  frame=none, % adds a frame around the code
%  frame=single,
  tabsize=4, % sets default tabsize to 2 spaces
  breaklines=true, % sets automatic line breaking
  breakatwhitespace=false, % sets if automatic breaks should only happen at whitespace
  % also try caption instead of title
  escapeinside={\%*}{*)}, % if you want to add a comment within your code
  morekeywords={*,...}, % if you want to add more keywords to the set
  keywordstyle=\color[rgb]{0,0,1},
  commentstyle=\color[rgb]{0.133,0.545,0.133}\textit,
  stringstyle=\color[rgb]{0.627,0.126,0.941},
}}


%default configuration for lstset, so lstinline fragments are displayed correctly before the first lstlisting appears
\lstset{ %
  language=C++, % choose the language of the code
  basicstyle=\footnotesize\ttfamily, % the size of the fonts that are used for the code
  numbers=none, % where to put the line-numbers
  numberstyle=\footnotesize\ttfamily\color[rgb]{0.6,0.6,0.6}, % the size of the fonts that are used for the line-numbers
  stepnumber=1, % the step between two line-numbers. If it's 1 each line
  xleftmargin=4mm,
  % will be numbered
  numbersep=5pt, % how far the line-numbers are from the code
  backgroundcolor=\color{white}, % choose the background color. You must add \usepackage{color}
  showspaces=false, % show spaces adding particular underscores
  showstringspaces=false, % underline spaces within strings
  showtabs=false, % show tabs within strings adding particular underscores
  frame=none, % adds a frame around the code
%  frame=single,
  tabsize=4, % sets default tabsize to 2 spaces
  breaklines=true, % sets automatic line breaking
  breakatwhitespace=false, % sets if automatic breaks should only happen at whitespace
  % also try caption instead of title
  escapeinside={\%*}{*)}, % if you want to add a comment within your code
  morekeywords={*,...}, % if you want to add more keywords to the set
  keywordstyle=\color[rgb]{0,0,1},
  commentstyle=\color[rgb]{0.133,0.545,0.133}\textit,
  stringstyle=\color[rgb]{0.627,0.126,0.941},
}


%newcommands for exercises and solutions
\newcounter{excounter} \setcounter{excounter}{0}
\newcounter{partexcounter} \setcounter{partexcounter}{0}
\newcommand{\exercise}[1]{\stepcounter{excounter}\setcounter{partexcounter}{0}\vspace*{10px} \noindent \textbf{\textsf{Exercise \arabic{excounter})\quad #1}}\quad \\}
\newcommand{\partexercise}{\stepcounter{partexcounter}\noindent \textbf{\textsf{(\alph{partexcounter})}}\quad}

\newcounter{partsolcounter} \setcounter{partsolcounter}{0}
\newcommand{\solution}[1]{\vspace*{10px} \noindent \textbf{\textsf{Solution \arabic{excounter})}} \quad \textit{#1}\\}
\newcommand{\partsolution}{\stepcounter{partsolcounter} \noindent \textbf{\textsf{(\alph{partsolcounter})}}\\}


%reminder to answer students' questions
\newcommand{\quest}{\begin{center}
                    \Large{\textit{Questions?}}
                    \end{center}
}


% header and footer definitions if someone wants to introduce headers and footers to the sheets
%
%\usepackage{fancyhdr}
%\pagestyle{fancy}
%\fancyhf{}
%
% upper right
%\fancyhead[R]{\textsf{Exercise something}}
%
% upper left 
%\fancyhead[L]{whatever}
%
% upper middle
%\fancyhead[C]{\textbf{Exercise Class 1}}
%
% lower middle
%\fancyfoot[C]{\textsf{Seite \thepage}}
%
% lower left 
%\fancyfoot[L]{\textsf{\today}}
%
% lower right
%\fancyfoot[R]{\textsf{blabla}}

\newcommand{\strong}[1]{\textbf{#1}}
\newcommand{\slide}[1]{\textbf{slide #1}}
\newcommand{\critical}[1]{\textcolor{red}{\textbf{#1}}}
\newcommand{\TODO}[1]{\textbf{TODO: #1}}
\newcommand{\Expert}{[code]expert}
\newcommand{\Codeboard}{Codeboard}
\newcommand{\ExerciseClass}[1]{%
\begin{center}
	\LARGE{\textbf{\textsf{Exercise Class \the\numexpr #1 \relax}}}
\end{center}}

\newcommand{\blackboard}{blackboard}
\newcommand{\slides}{slides}
\newcommand{\ide}{programming environment}
\newcommand{\PreparationSection}[1]{\section{\textcolor{red}{Preparation: #1}}}
\newcommand{\ImportantSection}[2]{\section{\textcolor{red}{#1} (#2 min.)}}
\newcommand{\MandatorySection}[2]{\section{\textcolor{red}{#1} (#2 min.)}}
\newcommand{\OptionalSection}[2]{\section{#1 (#2 min.)}}
\newcommand{\AdditionalMaterialSection}[2]{%
\marginnote{%
    The additional material sections provide some additional classroom
    activities which you can use in case you decide that the provided
    lesson plan does not work for you.
}[1cm]
\section{Additional Material: #1 (#2 min.)}}
\newcommand{\SubSectionInformational}[1]{\subsection{#1}}
\newcommand{\SubSection}[1]{\subsection{#1}}
\newcommand{\SubSectionWith}[2]{\subsection{#1 (#2)}}
\newcommand{\SubSectionWithTime}[3]{\subsection{#1 (#2 min., #3)}}
\newcommand{\SubSectionExplanation}[2]{\subsection{Explanation (#1 min., #2)}}
\newcommand{\SubSectionExplanationOptional}[2]{\subsection{(Optional) Explanation (#1 min., #2)}}
\newcommand{\SubSectionExercise}[2]{\subsection{Exercise: #1 (#2 min.)}}
\newcommand{\SubSectionExerciseTask}[2]{\subsubsection{Task (#1 min., #2)}}
\newcommand{\SubSectionExerciseThink}[1]{\subsubsection{Reading Task and Individual Thinking (#1 min.)}}
\newcommand{\SubSectionExercisePair}[1]{\subsubsection{Pair Programming (#1 min., \ide{})}}
\newcommand{\SubSectionExerciseIndividual}[1]{\subsubsection{Individual Programming (#1 min., \ide{})}}
\newcommand{\SubSectionExercisePairDiscuss}[1]{\subsubsection{Pair Discussion (#1 min.)}}
\newcommand{\SubSectionExercisePairDiscussShare}[1]{\subsubsection{Pair Programming, Discussion and Share Solution (#1 min.)}}
\newcommand{\SubSectionExerciseShare}[1]{\subsubsection{Share Solution (#1 min., \ide{})}}
\newcommand{\SubSectionExerciseShareSolution}[2]{\subsubsection{Share Solution (#1 min., #2)}}
\newcommand{\SubSectionExerciseSolution}[2]{\subsubsection{Solution (#1 min., #2)}}

\newcommand{\listofpreparationtasks}{Prepare Before the Class}
\newlistof{preparationtask}{lex}{\listofpreparationtasks}

\mdfdefinestyle{Preparation}{%
frametitle={Preparation},
backgroundcolor=black!10,
bottomline=false,
topline=false,
rightline=false,
leftline=false
}
\newenvironment{Preparation}{%
\newcommand{\PreparationBeforeClass}[1]{%
\addcontentsline{lex}{preparationtask}{$\square$ ##1}%
$\square$ ##1\\}
\newcommand{\PreparationInClass}[1]{$\square$ ##1\\}
\begin{mdframed}[style=Preparation]%
}{%
\end{mdframed}%
}

\mdfdefinestyle{Question}{%
frametitle={Question},
}
\newenvironment{Question}{%
\newenvironment{Answer}{\hfill\\[1em]\strong{Possible Answer}\\}{}
\newenvironment{OnlyAnswer}{\hfill\\[1em]\strong{Answer}\\}{}
\begin{mdframed}[style=Question]%
}{%
\end{mdframed}%
}


\mdfdefinestyle{Explanation}{%
backgroundcolor=black!5,
}
\newenvironment{Explanation}{%
\begin{mdframed}[style=Explanation]%
}{%
\end{mdframed}%
}

\newenvironment{Theory}{%
\section*{Theory}

The theory points of this exercise session are:
\begin{enumerate}}{\end{enumerate}}

\newenvironment{SeenInClass}{%
\section*{Seen in class}
During the last class, students have seen:
}



% Some commands copied from the exam.
\newcommand{\rmin}{\mathrm{min}}
\newcommand{\rmax}{\mathrm{max}}
\newcommand{\boxx}[3]{\begin{minipage}[c][#3]{#2}#1~ \end{minipage}}



\begin{document}

\ExerciseClass{1}

\begin{Theory}
\item Elastic Membranes
\item Electrostatic Fields
\item Quadratic Minimization Problems
\item (Will be done in Week 2) Sobolev spaces
\item (Will be done in Week 2) Linear Variational Problems
\end{Theory}

\section*{Exercises}

The exercises reviewed in this exercise class are:
\begin{itemize}
    \item Problem 0-1: Setting up implementation environment for NumPDE course (41 min)
    \item Problem 0-2: Using the GDB Debugger (80 min)
    \item Problem 1-1: Quadratic Functionals, only sub-problems a) b) d) g) (50 min)
\end{itemize}

% \listofpreparationtask

\textbf{Recommendation:} Keep 1,2 and 3 under 30 minutes as to be able to make them have a working installation by the end of the exercise session.

\tableofcontents

\newpage

\OptionalSection{Introduction}{20}

\SubSectionWith{Introduce Yourself}{1min}

\url{https://polybox.ethz.ch/index.php/s/qhIJVzhSKtYg3QY/download}.

\SubSectionWith{Introduce the topic}{9min}

Choose one or two example PDEs and its applications in the real world as well as a live vizualisation. Here are two basic examples:
\begin{enumerate}

\item 
\begin{enumerate}
    \item Prepare by showing a picture of the fur of a cat and having the reaction diffusion equation written on the blackboard. Ask the students about the relation between both.
    \item Introduce a reaction system with diffusion: $$(\xrightarrow{f(1-u)} u) + 2v \xleftrightarrow{} 2v + (v \xrightarrow{k+f})$$ i.e u has feed rate f(1-u), v kill rate k+f, and one u and two v reacts to form 3 v. u and v are chemicals diffusing in some liquid solution.
    \item Relates it to the reaction Diffusion Equation. Give the intuition for diffusion.
    \begin{equation}
     \pdv{u}{t} = {D}_{1}\Delta u - uv^2 + f(1-u)
     \quad
     \pdv{v}{t} = {D}_{2}\Delta u + uv^2 - (k+f)v
    \end{equation}
    \item Show vizualization: \url{https://n.ethz.ch/~samrusso/#/algo}
\end{enumerate}

\item
\begin{enumerate}
    \item Prepare by showing a picture of a plane or a rocket launch and having the incompressible euler equation written on the blackboard. Ask the students about the relation between both.
    \item Briefly introduce the incompressible euler equation $\pdv{u}{t} + u \cdot \nabla{u} + \nabla{p} = 0, \quad \nabla \cdot u = 0$. You can for example give the intuition about advection and how it relates to the PDE.
    \item Show vizaluation \url{https://n.ethz.ch/~samrusso/#/algo}
\end{enumerate}

\end{enumerate}



\SubSectionWith{Get to Know Your Students (optional)}{10min}

Put some get-to-know basic question (what is your OS, what are you expactations for this your course, what was your favorite course so far, what is your favorite field of application for CSE, ...) on the board and make every student answer it. It might seems like a "waste of time" but making every student speak at least once in their first exercise session will get them used to speaking in this particular and is important to break the ice\footnote{This learning activity is called “Whip (Around)” and
it was taken from Kimberly D. Tanner paper “Structure Matters:
Twenty-One Teaching Strategies to Promote Student Engagement and
Cultivate Classroom Equity”
(\url{https://polybox.ethz.ch/index.php/s/sSwgHBCQgihxC3m/download}).}.

\MandatorySection{Administrative Matters}{10}

\SubSection{General infos}

\begin{quote}
    - Course structure: one exercise class, one Q\&A, online videos, lecture notes (review questions \url{https://exams.vis.ethz.ch/exams/u3qv50iv.pdf}), problem notes (come from old exams), script. Explain them what was your way of dealing with the different material during the time you took NumPDE. Show them the Discuna platform and how they can directly link their question to the document.
    
    - Semester-Exams: Explain the importance of preparing well for the mid term and end term. No helping sheets allowed. Good as it forces you to keep the pace with the course.
    
    - Final-Exam: Explain the structure of the final exam: (30'-45') reading time, 3 hours written, full script available, no helping sheets allowed.
\end{quote}

\quest

\SubSection{Exercise Sessions}

\begin{quote}
    - Explain them how they can hand-in their exercises: Moodle, email or GitHub issues.
    - Tell them how you will structure your exercise class (theory, exercises, etc.)
    - Set the environment for them to feel free to ask questions
    
    (e.g., directly from the intro to C++ lesson plan)

    Your active participation is important! We plan to have mostly
    interactive classroom activities and constructive discussion during
    the exercise sessions.

    As a result, we expect that you will actively participate by saying
    your opinion, educated guesses, and asking questions.

    Please note that since we are learning we expect you to make
    mistakes. Therefore, do not be afraid to share your educated guesses
    when you do not know the correct answer. It is much better to try to
    participate, make mistakes, and learn from them than to passively
    listen.
\end{quote}

\quest

\OptionalSection{Quadratic Minimization Problems}{20}

\SubSectionWith{Introduction}{15min} 

\begin{enumerate}
    \item 
Introduce your favorite energy potential, e.g. 1d elastic string under vertical loading
\begin{equation}
    J_s(u) = \int_a^b \frac{1}{2}\sigma(x) (\frac{du}{dx}(x))^2 - f(x)u(x) dx
\end{equation}

From Hook's law on strain energy on an rod with nodes $x_i$:
\begin{equation}
    \epsilon = \frac{\Delta l_i}{L_i} = \frac{u(x_{i+1}) - u(x_i)}{x_{i+1} - x_i} \xrightarrow{\Delta x \rightarrow 0} \frac{du}{dx}(x)
\end{equation}
\begin{equation}
    E_{elastic} = \frac{1}{2}\sigma(x)\epsilon^2
\end{equation}

With $u(x) = x - x'$ being the displacement field from original to deformed state.

\item
Introduce Equilibrium principle: system tends to minimum energy. Ask to the students what is missing in this equation:
\begin{equation}
    u = argmin_v \ J_s(v)
\end{equation}

Configuration space is missing, introduce configuration space: space (=set) of functions that we want to search, depends on the problem we are solving. E.g., for 1d string we want it to be continuous:
\begin{equation}
    V = \{u \in C^0([a, b]), u(a) = u_0, u(b) = u_1\}
\end{equation}

Ask the students what is the problem with configuration space and the potential presented: we are asking for the derivative of u but our configuration space includes non-differentiable functions!
Thus we need to constrain our space more (draw example of $C^1_{pw}$ function).

\begin{equation}
    V = \{u \in C^1_{pw}([a,b]), u(a) = u_0, u(b) = u_1\}
\end{equation}

\item Introduce bilinear forms, p.d $a(v,v) > 0 \text{ and } a(0, 0) = 0$ and s.p.d $a(v,v) \geq 0$. For bilinear forms, give them the trick that the coefficients of $a(a+b, c+d)$ behave like the one of a multiplication $(a+b)*(c+d)$.

\end{enumerate}



\SubSectionWith{Example solving}{3min}

Solve subexercise 1.b. Show them how you can find a counter example: you want a second derivative that is not zero and a first derivative that is negative.

\MandatorySection{Setting up coding environment}{90min}

This might be a struggle for some students, so keep your patience and try to help them as best as you can. Don't hesitate to ask if someone had the same problem in class and try to make them cooperate and help each other as you might be easily overwhelmed otherwise.\\
Maybe suggest to fork repo instead of clone, to be able to push and add us to correct.

Recommandation of setup:
\begin{enumerate}
    \item CLion
    \item VSCode + CMake extension
\end{enumerate}

These setups work pretty well out of the box. The students just have to clone the repo and open the NPDECODES folder. They might need to configure they're CMake plugin in VSCode


\SubSection{Linux setup}

Setup for VSCode:
\begin{enumerate}
    \item Install extensions for vscode (c++, cmake tools, clangd (LSP, see below))
    \item activate CMAKE\_EXPORT\_COMPILE\_COMMANDS in the cmake configuration i.e. call `ccmake .` in
    the build folder (which is automatically created by cmake tools (maybe you need to build a file first)).
    Then go to advanced options (i.e. in the cmake configuration type t) search and enable the option.
    (Most likely the  LSP will not recognize the Lehrfem and Eigen library, with this option cmake will produce a file 
    called "compile\_commands.json" this file will contain the necessary information for clangd to find these libraries)
    \item Build any file with the cmake tools extension to rebuild the cmake configuration.    
    \item Use the CMake tools extension to debug files.
\end{enumerate}

Setup for CLion
\begin{enumerate}
    \item apply for student license (\url{https://www.jetbrains.com/shop/eform/students})
    \item open project (needs to be root folder, where base CMakeLists.txt is)
    \item choose debug
    \item CLion will do the rest for you
    \item in top right can choose target, (choose non static to run/debug)
\end{enumerate}

\SubSection{Windows Setup}
The students might not be able to install cmake or a compiler. Recommend them to use WSL or a VM to a Unix-based distribution. WSL is straightforward to install and more lightweight than an usual VM and works very well with VSCode and CLion. 

Typical setup for VSCode:
\begin{enumerate}
    \item Install Ubuntu through WSL through admin terminal: \$ wsl --install
    \item Install WSL and CMake extension in VSCode
    \item Open Ubuntu (e.g. from start menu) and git clone the NPDECODE repo. 
    \item To open the VSCode server, from Ubuntu NPDECODE repo: ../NPDECODES:\$ code . 
    \item From VSCode, the CMake extension will start its setup. Automatically uses WSL compiler.
\end{enumerate}

Typical errors and fixes:
\begin{itemize}
    \item No compilers installed in WSL -> make them install g++
\end{itemize}

For CLion:
\begin{enumerate}
    \item Install Ubuntu through WSL through admin terminal: \$ wsl --install
    \item Open Ubuntu (e.g. from start menu) and git clone the NPDECODE repo. 
    \item Open CLion, Ctrl+O, WSL folders will be visible, open NPDECODE repo. CLion will automatically start to setup the CMake. For the CMake config, select the WSL compiler.
\end{enumerate}

Typical errors and fixes:
\begin{itemize}
    \item
    WSL by default has only half of the RAM available. If the student computer only has 8GB of RAM, they might not be able to run CLion as it requires at least 4GB to run. One can change the default setting by creating a .wslconfig file in their user profile folder (can be access by windows+r -> \%UserProfile\%) and write in it:
    \begin{lstlisting}
    [wsl2]
    memory=8GB
    \end{lstlisting}
\end{itemize}





\SubSection{MacOS setup}


\end{document}
