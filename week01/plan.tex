%Header for the instruction files


%%%%%%%%%%%%%%%%%%%%%%%%%%%%%%%%%%%%%%%%%%%% PACKAGES %%%%%%%%%%%%%%%%%%%%%%%%%%%%%%%%%%%%%%%%%%%%%%%%%%%%%%%%%

%euopean version of article, because paper format is A4
\documentclass{scrartcl}

\usepackage[a4paper, left=5mm, right=10cm, top=1cm, marginparwidth=9cm]{geometry}

%UTF-8, because want decent encoding
\usepackage[utf8]{inputenc}

%\usepackage[ngerman]{babel}

%for ae ue oe
%\usepackage{ae}
\usepackage[T1]{fontenc}

%better monospaced font
\usepackage[scaled=0.8]{beramono}

%to have math symbols available
\usepackage{amsmath,amssymb,euscript}

%pretty enumerations
\usepackage{enumerate}

%pictures 
\usepackage{graphicx}
%\DeclareGraphicsRule{.tif}{png}{.png}{`convert #1 `dirname #1`/`basename #1 .tif`.png}

%listings are a nice way to display source code
\usepackage{listings, color}
\usepackage[table]{xcolor}

%define another version of green in addition to the 16 tex colors
\definecolor{mygreen}{rgb}{0.133,0.545,0.133}

%for weblinks
\usepackage[hidelinks]{hyperref}

\usepackage{mdframed}

\usepackage{tocloft}

\usepackage{marginnote}

\usepackage{tabularx}

\usepackage{ textcomp }

%%%%%%%%%%%%%%%%%%%%%%%%%%%%%%%%%%%%%%%%%%%% COMMANDS AND SETTINGS %%%%%%%%%%%%%%%%%%%%%%%%%%%%%%%%%%%%%%%%%%%%


%noindent, because text passages are very short to begin with
\setlength\parindent{0pt}

%instead the distance between lines is a bit bigger, if a paragraph ends
%this is a default value, it is sometimes changed locally, for example whenever itemize is used
%\setlength{\parskip}{1.0cm plus4mm minus3mm}

%use these commands when you want to create a list of points using itemize
%the distance between paragraphs defined above makes lists look very bad, use /nogap before creating a list and \gap after to restore settings
\newcommand{\nogap}{}%\setlength{\parskip}{0.0cm}}
\newcommand{\gap}{}%\setlength{\parskip}{1.0cm plus4mm minus3mm}}


%command that configures the code listings to be in a box with line numbers to the left
%it also makes sure that there is space between the last paragraph and the source code, so make sure you use this command before every 
%listing of a complete program
\newcommand{\numbersleft}{ 
\setlength{\parskip}{0cm}
{\scriptsize \mbox{}} \\ 
\setlength{\parskip}{1.0cm plus4mm minus3mm}
\lstset{ %
  language=C++, % choose the language of the code
  basicstyle=\footnotesize\ttfamily, % the size of the fonts that are used for the code
  numbers=left, % where to put the line-numbers
  numberstyle=\footnotesize\ttfamily\color[rgb]{0.6,0.6,0.6}, % the size of the fonts that are used for the line-numbers
  stepnumber=1, % the step between two line-numbers. If it's 1 each line
  xleftmargin=1mm,
  xrightmargin=1mm,
  % will be numbered
  numbersep=10pt, % how far the line-numbers are from the code
  backgroundcolor=\color{white}, % choose the background color. You must add \usepackage{color}
  showspaces=false, % show spaces adding particular underscores
  showstringspaces=false, % underline spaces within strings
  showtabs=false, % show tabs within strings adding particular underscores
  %frame=l, % adds a frame around the code
  frame=single,
  tabsize=4, % sets default tabsize to 2 spaces
  breaklines=true, % sets automatic line breaking
  breakatwhitespace=false, % sets if automatic breaks should only happen at whitespace
  % also try caption instead of title
  escapeinside={\%*}{*)}, % if you want to add a comment within your code
  morekeywords={*,...}, % if you want to add more keywords to the set
  keywordstyle=\color[rgb]{0,0,1},
  commentstyle=\color[rgb]{0.133,0.545,0.133}\textit,
  stringstyle=\color[rgb]{0.627,0.126,0.941},
}}

%command that configures the code listings to be without a box and without line numbers to the left
%it also makes sure that there is space between the last paragraph and the source code, so make sure you use this command before every listing which
%is just a code snippet
\newcommand{\nonumbers}{
\setlength{\parskip}{0cm}
{\scriptsize \mbox{}} \\ 
\setlength{\parskip}{1.0cm plus4mm minus3mm}
\lstset{ %
  language=C++, % choose the language of the code
  basicstyle=\footnotesize\ttfamily, % the size of the fonts that are used for the code
  numbers=none, % where to put the line-numbers
  numberstyle=\footnotesize\ttfamily\color[rgb]{0.6,0.6,0.6}, % the size of the fonts that are used for the line-numbers
  stepnumber=1, % the step between two line-numbers. If it's 1 each line
  xleftmargin=4mm,
  % will be numbered
  numbersep=5pt, % how far the line-numbers are from the code
  backgroundcolor=\color{white}, % choose the background color. You must add \usepackage{color}
  showspaces=false, % show spaces adding particular underscores
  showstringspaces=false, % underline spaces within strings
  showtabs=false, % show tabs within strings adding particular underscores
  frame=none, % adds a frame around the code
%  frame=single,
  tabsize=4, % sets default tabsize to 2 spaces
  breaklines=true, % sets automatic line breaking
  breakatwhitespace=false, % sets if automatic breaks should only happen at whitespace
  % also try caption instead of title
  escapeinside={\%*}{*)}, % if you want to add a comment within your code
  morekeywords={*,...}, % if you want to add more keywords to the set
  keywordstyle=\color[rgb]{0,0,1},
  commentstyle=\color[rgb]{0.133,0.545,0.133}\textit,
  stringstyle=\color[rgb]{0.627,0.126,0.941},
}}


%default configuration for lstset, so lstinline fragments are displayed correctly before the first lstlisting appears
\lstset{ %
  language=C++, % choose the language of the code
  basicstyle=\footnotesize\ttfamily, % the size of the fonts that are used for the code
  numbers=none, % where to put the line-numbers
  numberstyle=\footnotesize\ttfamily\color[rgb]{0.6,0.6,0.6}, % the size of the fonts that are used for the line-numbers
  stepnumber=1, % the step between two line-numbers. If it's 1 each line
  xleftmargin=4mm,
  % will be numbered
  numbersep=5pt, % how far the line-numbers are from the code
  backgroundcolor=\color{white}, % choose the background color. You must add \usepackage{color}
  showspaces=false, % show spaces adding particular underscores
  showstringspaces=false, % underline spaces within strings
  showtabs=false, % show tabs within strings adding particular underscores
  frame=none, % adds a frame around the code
%  frame=single,
  tabsize=4, % sets default tabsize to 2 spaces
  breaklines=true, % sets automatic line breaking
  breakatwhitespace=false, % sets if automatic breaks should only happen at whitespace
  % also try caption instead of title
  escapeinside={\%*}{*)}, % if you want to add a comment within your code
  morekeywords={*,...}, % if you want to add more keywords to the set
  keywordstyle=\color[rgb]{0,0,1},
  commentstyle=\color[rgb]{0.133,0.545,0.133}\textit,
  stringstyle=\color[rgb]{0.627,0.126,0.941},
}


%newcommands for exercises and solutions
\newcounter{excounter} \setcounter{excounter}{0}
\newcounter{partexcounter} \setcounter{partexcounter}{0}
\newcommand{\exercise}[1]{\stepcounter{excounter}\setcounter{partexcounter}{0}\vspace*{10px} \noindent \textbf{\textsf{Exercise \arabic{excounter})\quad #1}}\quad \\}
\newcommand{\partexercise}{\stepcounter{partexcounter}\noindent \textbf{\textsf{(\alph{partexcounter})}}\quad}

\newcounter{partsolcounter} \setcounter{partsolcounter}{0}
\newcommand{\solution}[1]{\vspace*{10px} \noindent \textbf{\textsf{Solution \arabic{excounter})}} \quad \textit{#1}\\}
\newcommand{\partsolution}{\stepcounter{partsolcounter} \noindent \textbf{\textsf{(\alph{partsolcounter})}}\\}


%reminder to answer students' questions
\newcommand{\quest}{\begin{center}
                    \Large{\textit{Questions?}}
                    \end{center}
}


% header and footer definitions if someone wants to introduce headers and footers to the sheets
%
%\usepackage{fancyhdr}
%\pagestyle{fancy}
%\fancyhf{}
%
% upper right
%\fancyhead[R]{\textsf{Exercise something}}
%
% upper left 
%\fancyhead[L]{whatever}
%
% upper middle
%\fancyhead[C]{\textbf{Exercise Class 1}}
%
% lower middle
%\fancyfoot[C]{\textsf{Seite \thepage}}
%
% lower left 
%\fancyfoot[L]{\textsf{\today}}
%
% lower right
%\fancyfoot[R]{\textsf{blabla}}

\newcommand{\strong}[1]{\textbf{#1}}
\newcommand{\slide}[1]{\textbf{slide #1}}
\newcommand{\critical}[1]{\textcolor{red}{\textbf{#1}}}
\newcommand{\TODO}[1]{\textbf{TODO: #1}}
\newcommand{\Expert}{[code]expert}
\newcommand{\Codeboard}{Codeboard}
\newcommand{\ExerciseClass}[1]{%
\begin{center}
	\LARGE{\textbf{\textsf{Exercise Class \the\numexpr #1 \relax}}}
\end{center}}

\newcommand{\blackboard}{blackboard}
\newcommand{\slides}{slides}
\newcommand{\ide}{programming environment}
\newcommand{\PreparationSection}[1]{\section{\textcolor{red}{Preparation: #1}}}
\newcommand{\ImportantSection}[2]{\section{\textcolor{red}{#1} (#2 min.)}}
\newcommand{\MandatorySection}[2]{\section{\textcolor{red}{#1} (#2 min.)}}
\newcommand{\OptionalSection}[2]{\section{#1 (#2 min.)}}
\newcommand{\AdditionalMaterialSection}[2]{%
\marginnote{%
    The additional material sections provide some additional classroom
    activities which you can use in case you decide that the provided
    lesson plan does not work for you.
}[1cm]
\section{Additional Material: #1 (#2 min.)}}
\newcommand{\SubSectionInformational}[1]{\subsection{#1}}
\newcommand{\SubSection}[1]{\subsection{#1}}
\newcommand{\SubSectionWith}[2]{\subsection{#1 (#2)}}
\newcommand{\SubSectionWithTime}[3]{\subsection{#1 (#2 min., #3)}}
\newcommand{\SubSectionExplanation}[2]{\subsection{Explanation (#1 min., #2)}}
\newcommand{\SubSectionExplanationOptional}[2]{\subsection{(Optional) Explanation (#1 min., #2)}}
\newcommand{\SubSectionExercise}[2]{\subsection{Exercise: #1 (#2 min.)}}
\newcommand{\SubSectionExerciseTask}[2]{\subsubsection{Task (#1 min., #2)}}
\newcommand{\SubSectionExerciseThink}[1]{\subsubsection{Reading Task and Individual Thinking (#1 min.)}}
\newcommand{\SubSectionExercisePair}[1]{\subsubsection{Pair Programming (#1 min., \ide{})}}
\newcommand{\SubSectionExerciseIndividual}[1]{\subsubsection{Individual Programming (#1 min., \ide{})}}
\newcommand{\SubSectionExercisePairDiscuss}[1]{\subsubsection{Pair Discussion (#1 min.)}}
\newcommand{\SubSectionExercisePairDiscussShare}[1]{\subsubsection{Pair Programming, Discussion and Share Solution (#1 min.)}}
\newcommand{\SubSectionExerciseShare}[1]{\subsubsection{Share Solution (#1 min., \ide{})}}
\newcommand{\SubSectionExerciseShareSolution}[2]{\subsubsection{Share Solution (#1 min., #2)}}
\newcommand{\SubSectionExerciseSolution}[2]{\subsubsection{Solution (#1 min., #2)}}

\newcommand{\listofpreparationtasks}{Prepare Before the Class}
\newlistof{preparationtask}{lex}{\listofpreparationtasks}

\mdfdefinestyle{Preparation}{%
frametitle={Preparation},
backgroundcolor=black!10,
bottomline=false,
topline=false,
rightline=false,
leftline=false
}
\newenvironment{Preparation}{%
\newcommand{\PreparationBeforeClass}[1]{%
\addcontentsline{lex}{preparationtask}{$\square$ ##1}%
$\square$ ##1\\}
\newcommand{\PreparationInClass}[1]{$\square$ ##1\\}
\begin{mdframed}[style=Preparation]%
}{%
\end{mdframed}%
}

\mdfdefinestyle{Question}{%
frametitle={Question},
}
\newenvironment{Question}{%
\newenvironment{Answer}{\hfill\\[1em]\strong{Possible Answer}\\}{}
\newenvironment{OnlyAnswer}{\hfill\\[1em]\strong{Answer}\\}{}
\begin{mdframed}[style=Question]%
}{%
\end{mdframed}%
}


\mdfdefinestyle{Explanation}{%
backgroundcolor=black!5,
}
\newenvironment{Explanation}{%
\begin{mdframed}[style=Explanation]%
}{%
\end{mdframed}%
}

\newenvironment{Objectives}{%
\section*{Objectives}

The objectives of this exercise session are:
\begin{enumerate}}{\end{enumerate}}

\newenvironment{SeenInClass}{%
\section*{Seen in class}
During the last class, students have seen:
}

% Some commands copied from the exam.
\newcommand{\rmin}{\mathrm{min}}
\newcommand{\rmax}{\mathrm{max}}
\newcommand{\boxx}[3]{\begin{minipage}[c][#3]{#2}#1~ \end{minipage}}


\begin{document}

\ExerciseClass{1}

\marginnote{%
The objectives section lists the goals we aim to achieve with this
exercise class. It is mostly up to you, teaching assistants, to decide
how you are going to achieve these goals. To save your time, we have
prepared a lesson plan with classroom activities that aim to achieve the
listed goals. You can choose whether you are going to use the provided
activities or design your own, whether you are going to follow the given
order or do the activities in a different one. In rare cases, the lesson
plan will contain activities, which you have to do (for example,
self-assessments); they are marked in red.}[1cm]

\begin{Objectives}
\item \label{o:welcome}
    Create a welcoming atmosphere for your students.
\item \label{o:administrative}
    Explain to students how the course is organised.
\item \label{o:expert}
    Explain to students how to use \Expert{}.
\item \label{o:assess}
    Assess the prior programming knowledge of your students.
\end{Objectives}

\marginnote{%
The lesson plan contains a large margin which you can use for your
notes. If you know \LaTeX, you can also edit the document directly by
using the GitLab web IDE. The source code is available at
\url{https://gitlab.inf.ethz.ch/OU-LECTURERS/teaching/computer-science-introduction/}.
GitLab should automatically build the PDF each time you commit your
changes. If you have any questions, do not hesitate to contact
the head TA.}[2em]
\listofpreparationtask

\marginnote{The time in parenthesis gives an estimate how long a
particular section should take. When reviewing the instructions, please
check if the estimate makes sense to you. This white space on the side
can be used for your notes.}[3.5cm]
\tableofcontents

\newpage

\OptionalSection{Introduction}{20}

\marginnote{%
The text in grey boxes are directed at you, teaching assistants. They
provide checklists and additional explanations.}[0.5cm]

\begin{Preparation}
    \PreparationBeforeClass{%
        Download the slides with the organisational information.}
    \PreparationBeforeClass{%
        Check early enough if the beamer works and the slides and
        programming environment are properly visible.}
\end{Preparation}

\SubSectionWithTime{Introduce Yourself}{2}{\slides}

Welcome your students. Show the \slide{2} and briefly introduce yourself
to your class\footnote{If you are interested you can find some advice
how to start the first class in the following handout:
\url{https://polybox.ethz.ch/index.php/s/qhIJVzhSKtYg3QY/download}.}:
\begin{enumerate}
    \item
        Say your name.
    \item
        Say what are you studying and why it is interesting to you (in
        simple words!).
\end{enumerate}

Also say what programming experience you have and why do you think this
course is important.

\SubSectionWithTime{Get to Know Your Students}{15}{\slides}

Keep the same slide with questions (\slide{2}) and ask each student to
very briefly answer both of them (30 seconds per
student)\footnote{This learning activity is called “Whip (Around)” and
it was taken from Kimberly D. Tanner paper “Structure Matters:
Twenty-One Teaching Strategies to Promote Student Engagement and
Cultivate Classroom Equity”
(\url{https://polybox.ethz.ch/index.php/s/sSwgHBCQgihxC3m/download}).}.

\MandatorySection{Administrative Matters}{15}

\SubSectionWith{High Level Course Organisation}{\slides}

\marginnote{%
The quoted text is the text that you should tell to students \strong{in
your own words}.}[1cm]

\slide{3}:
\begin{quote}
    The primary objective of this course is to learn programming C++.

    We are going to check whether you achieved this objective in the
    exam during the winter examination session.

    To perform well in the exam, you need to be able to program and to
    deal with the exam environment.

    To help you to learn to deal with both of them, we provide several
    “services”: lectures, weekly and bonus exercises, exercise sessions,
    self-assessments and StudyCenter.
\end{quote}

\quest

\SubSectionWith{Weekly Schedule}{\slides}

\slide{4}:
\begin{quote}
    New exercises are released every Monday.

    Then, there is a lecture covering the material necessary to solve
    the exercises.

    We are going to practice this material in the first exercise session
    that follows the lecture.

    If you have questions about the material, you can come to StudyCenter 
    (check the website on which days it is offered) to ask questions.

    You have ["one week" if IFMP/CSE else "two weeks"] to solve the exercises. Please note that the
    deadline is on Monday 18:00 and that late submissions are not accepted.

    After the exercise submission deadline, I will provide feedback to
    your weekly exercise submissions. We will discuss the most common
    misconceptions in the exercise session that is going to be one week
    after the submission deadline.
\end{quote}

\quest

\SubSectionWith{Weekly and Bonus Exercises}{\slides}

\slide{5}:
\begin{quote}
    All exercises will be managed in our online platform \Expert{}.

    There are two types of exercises. The first type of exercises is
    weekly exercises. The tasks of weekly exercises are usually very
    small and the purpose of them is to practice the material that was
    given in the lecture. By solving these exercises you will earn
    experience points.

    By earning enough experience points, you can unlock bonus exercises.
    On \Expert{}, you can see which weekly exercises give you experience
    points for which bonus tasks. As a rule of thumb, you will need to
    obtain about two thirds of all possible experience points to unlock
    the bonus exercise.

    The purpose of bonus exercises is to practice combining knowledge
    from different topics. There are going to be three bonus exercises.
    Obtaining 2/3 of possible points from bonus exercises allows earning
    a maximum bonus of 0.25 towards your final grade.

    It is important to remember that for solving exercises, it is 
    sufficient to use the constructs that were introduced in the course. 
    Furthermore, pay attention not to use constructs that are forbidden 
    in the task description. Note that the autograder might impose 
    further restrictions (such as not using global variables) that might 
    not be explicitly stated in the task description. Further note that 
    warnings are treated as errors. Thus, check the output of the autograder 
    carefully to avoid getting 0 points.
\end{quote}

\quest

\SubSectionWith{Exercise Sessions}{\slides}

\slide{7}:
\begin{quote}
    The purpose of the exercise sessions is to prepare you for solving
    future and past exercises.

    This means that during the exercise sessions we will revise some of
    the topics you need to solve future exercises and discuss common
    misunderstandings of past exercises.

    Your active participation is important! We plan to have mostly
    interactive classroom activities and constructive discussion during
    the exercise sessions.

    As a result, we expect that you will actively participate by saying
    your opinion, educated guesses, and asking questions.

    Please note that since we are learning we expect you to make
    mistakes. Therefore, do not be afraid to share your educated guesses
    when you do not know the correct answer. It is much better to try to
    participate, make mistakes, and learn from them than to passively
    listen.
\end{quote}

\quest

\begin{Explanation}
    If you cannot answer a particular question you got from students,
    feel free to forward it to the head TA or lecturers.
\end{Explanation}

\SubSectionWith{StudyCenter}{\slides}

\slide{8}:
\begin{quote}
    The student association organizes so called StudyCenter. The idea of
    this StudyCenter is that you can come and ask TAs questions which
    you did not have a chance to ask during the exercise session.
\end{quote}

\quest

\SubSectionWith{Self-Assessments}{\slides}

\slide{9}:
\begin{quote}
    The purpose of self-assessments is to let you to get familiar with
    the exam format and to get a chance to see where you have gaps.

    There are going to be five self-assessments and some of them will
    take place at the beginning of the exercise sessions.

    Today we will have a pre-course self-assessment. Its purpose is to
    better know your prior knowledge so that we can optimize the course
    for you, so no reason for you to worry about it.

    Next week, we will have the first real self-assessment. It will
    cover the material from the C++ tutorial, which you can find on the
    course website.
\end{quote}

\quest

\SubSectionWith{Info and Contact}{\slides}

\slide{10}:
\begin{quote}
    All information I presented can be found on the course website. More
    specifically, please read the organisational information sheet.

    If you have any questions about the course content, you can either
    ask during the class or in the StudyCenter.

    If you have any questions about exercises, ask me.

    If you have any administrative questions, ask the head assistant.
\end{quote}

\quest

\OptionalSection{How to use \Expert{}}{10}

\begin{Explanation}
    The goal of this activity is to ensure that all students can access
    \Expert{} and know how to use it.
\end{Explanation}

\SubSectionWithTime{Hello World}{10}{\ide}

\begin{Preparation}
    \PreparationBeforeClass{%
        Check that you have internet and can access \Expert{}.}
    \PreparationBeforeClass{%
        Make sure that you do not show the completed tasks of the first
        exercise. You can delete your existing code by clicking the
        trash bin icon next to the exercise.}
\end{Preparation}

Ask students to take out their laptops and follow you (do not forget to
ask if everyone successfully completed the step):
\begin{enumerate}
    \item Open \url{https://expert.ethz.ch} and navigate to the "Exercises" 
        tab.
    \item
        Click on “Task 1: Hello World”, this will open the
        task in the IDE.
    \item
        Show that the students can show/hide the task description by
        clicking on “Task”.
    \item
        Click on “hello.cpp” to open the file for editing.
    \item
        Ask students to dictate what you should write.
    \item
        Show that the students can compile the program by clicking on a
        green gear icon.
    \item
        Show that the students can run the program by clicking on a
        green triangle icon.
    \item
        Show that the students can run our tests by clicking on a green
        flask icon.
    \item
        Show that the students can submit their solution by clicking on
        “Create new submission”. Tell the students that you will provide
        feedback about their best submission and that they should not
        forget to submit.
    \item
        \strong{Important (new!):} explain to students that their solutions
        must pass all tests in order for them to get XP.
    \item
        \strong{Important (new!):} also say to the students that in the
        exam only compiling programs will be accepted and that we are
        going to use different tests for determining the final grade.
    \item
        \strong{Important:} show to students that the auto-grader can be
        sometimes very picky. For example, that it does not accept the
        program if the exclamation mark (“!”) is missing.
\end{enumerate}

\quest

Remind to students that at home they should do the C++ tutorial linked
on the website.

\MandatorySection{Self-Assessment 0}{35}

\begin{Preparation}
    \PreparationBeforeClass{%
        Check that you can access Self-Assessment 0.}
\end{Preparation}

\begin{Explanation}
    Since this is an introductory programming course, the students are
    not required to have any prior programming experience. Therefore,
    try to make sure that students who do not have any prior programming
    experience do not feel demotivated.
\end{Explanation}

\strong{Before} starting with the self-assessment, please tell students
the following:

\begin{quote}
    We know that some of you learned to program in school. However, we
    neither know how many of you learned programming, nor at which
    level. Therefore, to make the course more relevant and interesting,
    we would like learn about your programming experience.

    The sole purpose of this assessment is to improve the course and,
    therefore, your performance is not going to have any consequences
    for you. Also, if you have no experience in programming, do not
    worry; this is expected. However, even if you have no experience, we
    still encourage you to at least try to guess the answers.
\end{quote}

Ask students to open the self-assessment and start working on it.

\strong{After} the self assessment, please tell students the following:

\begin{quote}
    If you are interested, you can see the master solution of the
    self-assessment in Moodle directly. During this course you will
    learn all the concepts tested in this assessment and many more.

    To even out the differences in your background knowledge, we have
    prepared a tutorial that covers the basic concepts, which you will
    find on the course website. You should go over it until the next
    week when we are going to have the first real self-assessment. That
    self-assessment will cover the material from the tutorial.
\end{quote}

\OptionalSection{Program Tracing with Python Tutor}{10}

\begin{Explanation}
    Python Tutor is a tool that allows visualising program execution.
    This is useful for beginners who are struggling to understand the
    language semantics. Unfortunately, the service sometimes experiences
    a downtime. If this happens during your exercise session, just do
    the exercise entire on the blackboard and point out that such
    service exists which they can use for checking their solutions.
\end{Explanation}

Either write or display via projector the following program on the
blackboard (this program corresponds to question 6 from self-assessment
0):
\begin{lstlisting}
int main() {
  int x = 3;
  int y = 8;
  int z;
  z = x;
  x = y;
  y = z;
  return 0;
}
\end{lstlisting}

On the blackboard, draw a table with local variables:

\begin{tabular}{|l|p{2cm}|}
\hline
x &  \\ \hline
y &  \\ \hline
z &  \\ \hline
\end{tabular}

Together with students, execute the program step by step and fill in the
variable values into the table similarly to the Python Tutor
visualisation. However, instead of erasing the old value it is better to
cross it out so that students can still see the old value.

Show to students that they can use Python Tutor to check if they
executed the program correctly:

\url{http://pythontutor.com/cpp.html#mode=display}

Direct link to the example:

\url{http://pythontutor.com/cpp.html#code=int%20main%28%29%20%7B%0A%20%20int%20x%20%3D%203%3B%0A%20%20int%20y%20%3D%208%3B%0A%20%20int%20z%3B%0A%20%20z%20%3D%20x%3B%0A%20%20x%20%3D%20y%3B%0A%20%20y%20%3D%20z%3B%0A%20%20return%200%3B%0A%7D&curInstr=0&mode=display&origin=opt-frontend.js&py=cpp&rawInputLstJSON=%5B%5D}

\end{document}
